\Transcb{yellow}{blue}{Problems with Newtonian gravity}
\onecolumn
\begin{itemize}
\item Consider two pointmasses $m_{1}$ and $m_{2}$ at resp. locations $\vec{r}_{1}$ and $\vec{r}_{2}$
\item[] Gravity as described by Newton~:
\item[] Gravitational force $\vec{F}_{12}$ of $m_{1}$ exerted on $m_{2}$ at time $t$
\item[] $\displaystyle \qquad |\vec{F}_{12}|=\frac{Gm_{1}m_{2}}{|\vec{r}_{1}(t)-\vec{r}_{2}(t)|^{2}}
        =\frac{Gm_{1}m_{2}}{(\Delta r)^{2}}$ with $\Delta t=0$
\item[$\ast$] {\blue Clash with relativity} : $\Delta r$ and $\Delta t$ are no absolute values
\item[] $\rightarrow$ The above Newtonian formula is valid in only 1 inertial frame
\item[$\ast$] {\red Laws of physics should be identical in all inertial frames}
\item[] $\rightarrow$ {\blue New theory of gravity needed which is consistent with relativity}
\item Another puzzle~: $\displaystyle \vec{F}=\frac{\d\vec{p}}{\d t}=(\text{constant inertial mass}~m_{I})
                        =m_{I} \cdot \vec{a}$
\item[] At the Earth surface $|\vec{a}|$=constant and
        $|\vec{F}_{grav}|=m_{G}\cdot\frac{GM_{\oplus}}{R_{\oplus}^{2}}=m_{I}\cdot|\vec{a}|$
\item[] Why is {\blue gravitational mass $m_{G}$} equal to the {\blue inertial mass $m_{I}$} ?
\item[] Or~: How does gravity "know" how strong to pull to give all objects the same $\vec{a}$ ? 
\end{itemize}
